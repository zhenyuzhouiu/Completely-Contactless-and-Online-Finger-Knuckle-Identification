\subsection{3D Finger Knuckle Images Database}
The 3D finger knuckle images database \cite{3dfingerknuckle} can offer robust 3D information which can be invariant to changed illuminations, for example, the depth information of the crease of finger knuckle. With the 3D finger knuckle database, the FKNet is the state-of-the-art. Meanwhile, RFNet with TRTL loss function can get the best performance on the within database experiments and cross database experiments when compare to the FKNet on the 2D finger knuckle database. Therefore, we compare the RFNet with FKNet on the database to show the identification performance on 3D finger knuckle database. As for the protocol, it will generate $190*6 = 1140$ genuine matching scores, and $190*189*6=215,460$ imposter matching scores from matching matrix. From the Figure \ref{3dof3d-one-session}, RFNet-TRTL still can get the best performance for finger knuckle verification and identification. Form the ROC curve, the EER of the RFNet-TRTL can increase to $1.05\%$ while the EER of the FKNet is $2.4\%$. Not only on the 2D finger knuckle database, but also on the 3D finger knuckle database, the RFNet-TRTL can outperform the state-of-the-art results.

\begin{figure}[ht!]
    \centering
    \begin{subfigure}[b]{0.45\linewidth}
        \includegraphics[width=\linewidth]{Figures/3dod3d-roc_compare_new.eps}
        \caption{}
    \end{subfigure}
    \begin{subfigure}[b]{0.45\linewidth}
        \includegraphics[width=\linewidth]{Figures/3dod3d-cmc_compare_new.eps}
        \caption{}
    \end{subfigure}
    \caption{Comparative ROC (a) and corresponding CMC (b) for one-session of the 3D finger knuckle database \cite{3dfingerknuckle}.}
    \label{3dof3d-one-session}
\end{figure}


\subsection{Discussion}

\begin{figure}[ht!]
    \centering
    \begin{subfigure}[b]{\linewidth}
        \centering
        \includegraphics[width=0.15\linewidth]{Figures/EigenCAM/PolyUKnuckleV3/Origin/Session_1/104/1-0.jpg}
        \includegraphics[width=0.15\linewidth]{Figures/EigenCAM/PolyUKnuckleV3/Origin/Session_1/104/2-0.jpg}
        \includegraphics[width=0.15\linewidth]{Figures/EigenCAM/PolyUKnuckleV3/Origin/Session_1/104/3-0.jpg}
        \includegraphics[width=0.15\linewidth]{Figures/EigenCAM/PolyUKnuckleV3/Origin/Session_1/104/4-0.jpg}
        \includegraphics[width=0.15\linewidth]{Figures/EigenCAM/PolyUKnuckleV3/Origin/Session_1/104/5-0.jpg}
        \includegraphics[width=0.15\linewidth]{Figures/EigenCAM/PolyUKnuckleV3/Origin/Session_1/104/6-0.jpg}
        \caption{Input original finger knuckle images.}
    \end{subfigure}
    \begin{subfigure}[b]{\linewidth}
        \centering
        \includegraphics[width=0.15\linewidth]{Figures/EigenCAM/PolyUKnuckleV3/WRS/105-221/Session_1/104/1-0.jpg}
        \includegraphics[width=0.15\linewidth]{Figures/EigenCAM/PolyUKnuckleV3/WRS/105-221/Session_1/104/2-0.jpg}
        \includegraphics[width=0.15\linewidth]{Figures/EigenCAM/PolyUKnuckleV3/WRS/105-221/Session_1/104/3-0.jpg}
        \includegraphics[width=0.15\linewidth]{Figures/EigenCAM/PolyUKnuckleV3/WRS/105-221/Session_1/104/4-0.jpg}
        \includegraphics[width=0.15\linewidth]{Figures/EigenCAM/PolyUKnuckleV3/WRS/105-221/Session_1/104/5-0.jpg}
        \includegraphics[width=0.15\linewidth]{Figures/EigenCAM/PolyUKnuckleV3/WRS/105-221/Session_1/104/6-0.jpg}
        \caption{CAM of each finger knuckle output of RFNet trained with RSIL loss function.}
    \end{subfigure}
    \begin{subfigure}[b]{\linewidth}
        \centering
        \includegraphics[width=0.15\linewidth]{Figures/EigenCAM/PolyUKnuckleV3/WS/105-221/Session_1/104/1-0.jpg}
        \includegraphics[width=0.15\linewidth]{Figures/EigenCAM/PolyUKnuckleV3/WS/105-221/Session_1/104/2-0.jpg}
        \includegraphics[width=0.15\linewidth]{Figures/EigenCAM/PolyUKnuckleV3/WS/105-221/Session_1/104/3-0.jpg}
        \includegraphics[width=0.15\linewidth]{Figures/EigenCAM/PolyUKnuckleV3/WS/105-221/Session_1/104/4-0.jpg}
        \includegraphics[width=0.15\linewidth]{Figures/EigenCAM/PolyUKnuckleV3/WS/105-221/Session_1/104/5-0.jpg}
        \includegraphics[width=0.15\linewidth]{Figures/EigenCAM/PolyUKnuckleV3/WS/105-221/Session_1/104/6-0.jpg}
        \caption{CAM of each finger knuckle output of RFNet trained with SSTL loss function.}
    \end{subfigure}
    \caption{Show the class activation maps for the first session samples of 104th subject of the HKPolyU Finger Knuckle Images Database \cite{fingerknuckledbv3.0} with RFNet.}
\end{figure}

\begin{figure}[ht!]
    \centering
    \begin{subfigure}[b]{\linewidth}
        \centering
        \includegraphics[width=0.15\linewidth]{Figures/EigenCAM/PolyUKnuckleV3/Origin/Session_2/104/1-0.jpg}
        \includegraphics[width=0.15\linewidth]{Figures/EigenCAM/PolyUKnuckleV3/Origin/Session_2/104/2-0.jpg}
        \includegraphics[width=0.15\linewidth]{Figures/EigenCAM/PolyUKnuckleV3/Origin/Session_2/104/3-0.jpg}
        \includegraphics[width=0.15\linewidth]{Figures/EigenCAM/PolyUKnuckleV3/Origin/Session_2/104/4-0.jpg}
        \includegraphics[width=0.15\linewidth]{Figures/EigenCAM/PolyUKnuckleV3/Origin/Session_2/104/5-0.jpg}
        \includegraphics[width=0.15\linewidth]{Figures/EigenCAM/PolyUKnuckleV3/Origin/Session_2/104/6-0.jpg}
        \caption{Input original finger knuckle images.}
    \end{subfigure}
    \begin{subfigure}[b]{\linewidth}
        \centering
        \includegraphics[width=0.15\linewidth]{Figures/EigenCAM/PolyUKnuckleV3/WRS/105-221/Session_2/104/1-0.jpg}
        \includegraphics[width=0.15\linewidth]{Figures/EigenCAM/PolyUKnuckleV3/WRS/105-221/Session_2/104/2-0.jpg}
        \includegraphics[width=0.15\linewidth]{Figures/EigenCAM/PolyUKnuckleV3/WRS/105-221/Session_2/104/3-0.jpg}
        \includegraphics[width=0.15\linewidth]{Figures/EigenCAM/PolyUKnuckleV3/WRS/105-221/Session_2/104/4-0.jpg}
        \includegraphics[width=0.15\linewidth]{Figures/EigenCAM/PolyUKnuckleV3/WRS/105-221/Session_2/104/5-0.jpg}
        \includegraphics[width=0.15\linewidth]{Figures/EigenCAM/PolyUKnuckleV3/WRS/105-221/Session_2/104/6-0.jpg}
        \caption{CAM of each finger knuckle output of RFNet trained with RSIL loss function.}
    \end{subfigure}
    \begin{subfigure}[b]{\linewidth}
        \centering
        \includegraphics[width=0.15\linewidth]{Figures/EigenCAM/PolyUKnuckleV3/WS/105-221/Session_2/104/1-0.jpg}
        \includegraphics[width=0.15\linewidth]{Figures/EigenCAM/PolyUKnuckleV3/WS/105-221/Session_2/104/2-0.jpg}
        \includegraphics[width=0.15\linewidth]{Figures/EigenCAM/PolyUKnuckleV3/WS/105-221/Session_2/104/3-0.jpg}
        \includegraphics[width=0.15\linewidth]{Figures/EigenCAM/PolyUKnuckleV3/WS/105-221/Session_2/104/4-0.jpg}
        \includegraphics[width=0.15\linewidth]{Figures/EigenCAM/PolyUKnuckleV3/WS/105-221/Session_2/104/5-0.jpg}
        \includegraphics[width=0.15\linewidth]{Figures/EigenCAM/PolyUKnuckleV3/WS/105-221/Session_2/104/6-0.jpg}
        \caption{CAM of each finger knuckle output of RFNet trained with SSTL loss function.}
    \end{subfigure}
    \caption{Show the class activation maps (CAM) for the second session samples of 104th subject of the HKPolyU Finger Knuckle Images Database \cite{fingerknuckledbv3.0} with RFNet.}
\end{figure}


We have compared the identification performance of RFNet with EfficientNetV2-S, DeConvRFNet, and FKNet on the 2D and 3D finger knuckle database, and on the within database and cross database experiments. Due to deeply learned residual features of the RFNet which have already outperformed the state-of-the-art results on the palmprint, it still can get the best performance on the finger knuckle crease when compare to the rest models from above experiment results. Because finger knuckle is very prone to flexing, causing crease texture distortion, if just shift the template, it cannot solve the deformation problem with rotation. Therefore, we design our new TRTL to further solve the problem. With our TRTL loss function when train RFNet and MTRD when matching finger knuckle, the RFNet can increase matching accuracy regardless on the ROC and CMC based on the STTL. Especially on the Finger Knuckle Images Database (Version 3.0) which offer bending finger knuckle with complexity deformation, RFNet-TRTL improved performance is relatively more compare to other database from the Figure \ref{fkv3-one-session} and Figure \ref{fkv3-two-session}. Two-session protocol is more complexity because of changing finger knuckle crease and more complexity deformation when matching process. Form the Figure \ref{fkv3-two-session} (a) ROC and (b) CMC, our TRTL with RFNet get the best matching and recognition performance. Meanwhile, our TRTL not only work on the RFNet, but also can work on the DeConvRFNet from the Figure \ref{hd-one-session} and Figure \ref{2dof3d-one-session}, with TRTL loss function, the matching and recognition performance also can increase. 

From these experiment results, we can also get a conclusion is that EfficientV2-S model is better than FKNet from the within database experiment, and even on the Tsinghua finger knuckle database. EfficientNetV2 model can outperform the ResNet on the ImageNet \cite{russakovsky2015imagenet}, in other words, the EfficientNetV2 model can extract robust feature than ResNet on the ImageNet. Because EfficientNetV2 replace the residual block with inverted residual block, and  use MBConv as a block unit. As for MBConv block, it uses depth-wise convolutions to decrease training weights and use Squeeze-Excited block as channel attention. Meanwhile, the depth of EfficientNetV2-S is deeper than the FKNet. On the contrary, the FKNet use the ResNet-50 fist conv3 as the feature extract model. EfficientNetV2 use the more light, advance and efficient module than ResNet.

However, TRTL generalization ability is lower than STTL loss from the cross database experiment, except on the Tsinghua database. On the cross hand dorsal database, Figure \ref{crosshd-index-one-session}, EER of RFNet with TRTL performance will drop from about $2.0\%$ to $5.0\%$. As for the rest model, performance with TRTL also drop with corresponding value when compare to STTL. But in the within database experiment, these model with TRTL loss is better than STTL loss. It shows our TRTL can affect the back propagation during training process to the different model weights, in other words. And another phenomenon is that EfficientNetV2-S with SSTL and TRTL cannot work. From the Table \ref{model-complexity}, if the input image size is 300x300, EfficientNetV2-S with STTL and TRTL will output 9x9 template size. The output feature size is too small when use the STTL and TRTL loss function, inversely, the performance will drop while translation and rotation.


\begin{figure}[ht!]
    \centering
    \begin{subfigure}[b]{\linewidth}
        \centering
        \includegraphics[width=0.15\linewidth]{Figures/EigenCAM/PolyUHD/Origin/700/1.jpg}
        \includegraphics[width=0.15\linewidth]{Figures/EigenCAM/PolyUHD/Origin/700/2.jpg}
        \includegraphics[width=0.15\linewidth]{Figures/EigenCAM/PolyUHD/Origin/700/3.jpg}
        \includegraphics[width=0.15\linewidth]{Figures/EigenCAM/PolyUHD/Origin/700/4.jpg}
        \includegraphics[width=0.15\linewidth]{Figures/EigenCAM/PolyUHD/Origin/700/5.jpg}
        \caption{Input original images.}
    \end{subfigure}
    \begin{subfigure}[b]{\linewidth}
        \centering
        \includegraphics[width=0.15\linewidth]{Figures/EigenCAM/PolyUHD/WRS/700/1.jpg}
        \includegraphics[width=0.15\linewidth]{Figures/EigenCAM/PolyUHD/WRS/700/2.jpg}
        \includegraphics[width=0.15\linewidth]{Figures/EigenCAM/PolyUHD/WRS/700/3.jpg}
        \includegraphics[width=0.15\linewidth]{Figures/EigenCAM/PolyUHD/WRS/700/4.jpg}
        \includegraphics[width=0.15\linewidth]{Figures/EigenCAM/PolyUHD/WRS/700/5.jpg}
        \caption{RFNet trained with RSIL loss function.}
    \end{subfigure}
    \begin{subfigure}[b]{\linewidth}
        \centering
        \includegraphics[width=0.15\linewidth]{Figures/EigenCAM/PolyUHD/WS/700/1.jpg}
        \includegraphics[width=0.15\linewidth]{Figures/EigenCAM/PolyUHD/WS/700/2.jpg}
        \includegraphics[width=0.15\linewidth]{Figures/EigenCAM/PolyUHD/WS/700/3.jpg}
        \includegraphics[width=0.15\linewidth]{Figures/EigenCAM/PolyUHD/WS/700/4.jpg}
        \includegraphics[width=0.15\linewidth]{Figures/EigenCAM/PolyUHD/WS/700/5.jpg}
        \caption{RFNet trained with SSTL loss function.}
    \end{subfigure}
    \caption{Show the class activation maps for one subject samples of the HKPolyU Hand Dorsal Images Database \cite{ContactlessHnadDorsaldb} with RFNet.}
\end{figure}
