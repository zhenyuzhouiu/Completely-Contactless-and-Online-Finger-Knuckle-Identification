\section{Introduction}
Biometrics is the use of the human body's inherent physiological and behavioral characteristics for person identification. For the physiological characteristics, there are many human characteristics such as retina, iris, face, fingerprints and palm prints; as far as the behavioral characteristics, the available features include gait recognition, voice and other behavioral characteristics. As for the finger knuckle, it has also attracted many researchers devoted to it, because it is easier to expose, easier to collect, and it has also been proven to be unique and stable \cite{kumar2014importance}.


Many early works on biometric recognition of finger knuckle, ranging from coding-based methods, subspace methods and texture analysis methods to 3D shape patterns based on 3D image reconstruction and different methods have been used to achieve highly accurate recognition results. Most of the current finger knuckle biometric identification systems are based on the contact method. 

However, as for the contactless and online finger knuckle identification topic, it is a relatively new field, but should be studied extensively because it is more security and more hygienic, especially now with COVID-19. Although many methods have obtained good recognition accuracy, the finger knuckles are easily deformed under practical application scenarios, and the finger knuckle features will change accordingly. Thus, the matching accuracy will be degraded. Because of the above problems, there are corresponding studies to solve the finger knuckle deformation problem and provide new data sets and new methods \cite{kumar2019toward}. 


Moreover, in the actual application scenario, the data collected will be affected by various noisy backgrounds, different lighting, various camera focus and resolution, thus affecting the actual real application scenario. Of course, these problems can be solved to some extent. For example, to solve the difference of finger knuckle characteristics due to different lighting when taking pictures, FFT can be used to extract the frequency information from the pictures. For the deformation problem of finger knuckles, one can use adopt to select fixed key points and perform key points matching between corresponding images, use local feature descriptors for these key points to calculate the matching score, and finally, use the average matching score as the final matching score \cite{kumar2019toward}. In conclusion, if we want to perform contactless finger knuckle recognition in a real-world scenario, the most critical problem comes from two aspects: how to efficiently perform finger knuckle segmentation and correction, and the second one is how to match with high accuracy in real-world applications.

\section{Our Work}
Contactless finger knuckle matching research has not only received much input from researchers, but also there are a large number of different corresponding datasets available. This paper uses the major finger knuckle for contactless biometrics, and between proximal phalange and middle phalange are called major finger knuckle. From the beginning, we have taken the application scenario out of the experimental environment and jumped into the wild, which is a real contactless way in the real scenario. We also created our cross-platform finger knuckle recognition system software, which is no longer an experimental result on data, but a working recognition software. From object detection, ROI extraction to feature extraction, a series of traditional image processing algorithms are replaced by neural networks.

A new finger knuckle object detection algorithm is used, which can automatically extract the region of interest of finger knuckle based on the YOLOv5 \cite{YOLOv5} model framework and integrates the rotation prediction function of the target object prediction frame. The angle information of the finger knuckle can be obtained. A factor to be considered in the target matching algorithm is the rotation angle of the target object. Suppose the angle information of the target is not known. In that case, it is necessary to match multiple times within 360 degrees in the figure, which is not only time consuming, but also the matching accuracy decreases, so the angle regression based on the horizontal bounding box using CSL \cite{yang2020csl}, which in turn he angle information of the target object can be obtained. It is beneficial to the matching speed and accuracy of the matching algorithm and the automatic target segmentation using the object detection model to extract the ROI region of the major finger knuckle. The feature extraction step is also obtained based on the ResNet neural network model \cite{he2016deep}, and the matching algorithm uses the pixel-to-pixel minimum variance as the matching score. When calculating the pixel-to-pixel minimum variance, the pixel-to-pixel deviation can be compensated to some extent using pixel panning due to the offset, distortion and even the extraction of the ROI of the finger knuckles. In this paper, we have verified the theory and experimented with the whole process and created a cross-platform GUI based on the Qt Creator platform.

Chapter 2 will explain the public database of finger knuckles, the custom training data for finger knuckle detection models in the wild case, and the training database for feature extraction models. Chapter 3 introduces the finger knuckle detection model and how to implement the detection of finger knuckle angle information. For a complete biometric process, the segmentation is followed by the feature extraction process and the feature matching process, introduced in Chapter 4. Chapter 5 is the software production of finger knuckle recognition by connecting the whole process in series.


\section{Contactless Finger Knuckle Segmentation}

\subsection{Challenges of Finger Knuckle Segmentation}

As the most important module of a biometric system, the accuracy of the corresponding segmentation region of interest is crucial and affects the accuracy of the subsequent modules. A suitable ROI segmentation method can effectively improve the efficiency of matching and the accuracy of the matching algorithm. For the efficiency of the matching algorithm, there are two factors that can determine the matching efficiency, one is the size of the region of interest, and the other is the matching angle range. Whether local feature-based or holistic feature-based recognition algorithms, they both operate after the region of interest segmentation. The size of the region of interest should be comparable to the size of the actual target object at the pixel level so that the size obtained after segmentation will not have redundant pixel information, which will reduce the pixel values to be computed for both the extraction and the subsequent matching sessions. Since the target object has an angular rotation problem, the matching process generally requires simultaneous matching in multiple angular ranges, and the number of pixels to be computed increases exponentially. If the algorithm of ROI segmentation is accurate enough and the accuracy of the rotation is also high enough, this will naturally improve the detection efficiency. For the problem of improving the matching accuracy, if the accuracy of the region of interest is high enough, the background interference information extracted will be correspondingly more petite, and the pixel signal of the target object obtained is enough, the signal-to-noise ratio will be high, and the matching accuracy will be improved accordingly.

However, most of the current finger knuckle segmentation approaches are based on contact finger knuckle segmentation. Even for the contactless finger knuckle segmentation problem, their \cite{kumar2019toward}, \cite{cheng2012contactless} approach is to fix the finger knuckle position in the image when taking the finger knuckle data, and if the finger appears in the image in a different position or if there are multiple fingers, the problem of not detecting the finger knuckle or missing the finger knuckle is likely to occur. This segmentation is prone to errors, and this is not the only problem. Most importantly, the traditional segmentation algorithm cannot correctly segment the finger knuckles in the presence of complex background interference, multiple finger knuckles in the same field of view, obscured finger knuckles or bent finger knuckles. Since the subsequent operations of the finger knuckle recognition algorithm are based on the segmented image, the segmentation of finger knuckles affects the accuracy of finger knuckle recognition. It is vital to improving the efficiency of segmenting the finger knuckle region in any scenario.

\section{Contactless Finger Knuckle Detection}
This paper studies the finger knuckle region and uses the corresponding finger knuckle crease patterns as features of the finger knuckles. So the region of interest to be extracted is the part that can represent the skin crease patterns on the back of the finger. As mentioned in Section 3.1, it is difficult to use traditional object segmentation methods to automatically segment finger knuckles for applications such as in the wild. Even though there have been corresponding conventional algorithms implemented to automatically segment the finger knuckle region independent of the finger knuckle position pose \cite{kumar2019toward}, the method used in this paper requires fixing the position of the finger knuckles appearing in the image and is a Fixed ROI extraction. 

In order to solve the problem of finger knuckle detection in the real world, this paper chooses to use neural network models instead of traditional segmentation algorithms. Models of neural network models for object detection have achieved great success, whether it is the sliding window detection algorithm, the 2-stage series of R-CNN models \cite{girshick2014rich}, \cite{girshick2015fast}, \cite{ren2015faster}, or the 1-stage YOLO series \cite{redmon2016you}, \cite{redmon2017yolo9000}, \cite{redmon2018yolov3}, \cite{bochkovskiy2020yolov4} and SSD models \cite{liu2016ssd} up to the current position, and even the anchor-free based object detection algorithm \cite{xin2021pafnet} as well. Each of these models has its advantages. For the 2-stage model, the object detection accuracy is guaranteed, the 1-stage based model is a speedup based on the positive accuracy, and the anchor-free is a further improvement in the detection speed. In this paper, the latest version of the YOLO model series, YOLOv5 \cite{YOLOv5}, is used as the network model for finger knuckle detection because the YOLO series is famous for its fast detection speed and high accuracy. The module adopts various latest network modules, and the YOLOv5 model has a variety of model structures to meet different accuracy and speed requirements. For the YOLOv5 model, the number of layers of each submodule is varied to cope with different speed and accuracy requirements, while the overall structural component modules remain unchanged. The YOLOv5 neural network model used as a finger knuckle detection has good results in the wild, the prediction bounding boxes of YOLOv5 are based on a horizontal bounding box for regression. 


Although the results have been good for finger knuckle detection, they are not sufficient for the segmentation operation of finger knuckles. There are two main problems for rotating finger knuckles segmentation. The first problem is that when the horizontal bounding box is used to predict the object, the size of the bounding box is a minimum external horizontal rectangle for the size of the object. In such a case, when there is a rotation of the object, which is not horizontal for the picture, the horizontal box will have much more background for the detected object, and in the case that the target object is crowded, it is easy to eliminate the neighbouring of the horizontal bounding boxes are eliminated. As shown in Figure \ref{HBB Problems} (a), due to the rotation of the finger and the density of the finger, the prediction bounding box of the middle finger knuckle and the major finger knuckle of the ring finger have overlapped, so it is easy to be deleted during the non-max suppressing process, even if CIOU \cite{zheng2020distance} or DCOU \cite{zheng2021enhancing} is used. As shown in Figure \ref{HBB Problems} (b), after changing the threshold of IOU in the non-max suppressing process, the major finger knuckle of the ring finger is not detected.

The second problem is that for the segmentation operation if the horizontal box is used directly to carry out the segmentation operation, it is the same as shown in Figure \ref{HBB Problems} (c)-(i), which is the region of interest of the finger knuckle corresponding to the segmentation in Figure \ref{HBB Problems} (a). However, the segmentation to the background interference has a lot, and it is not considered a good finger knuckle segmentation operation. In order to solve the above problem, a rotated prediction bounding box was then used to deal with the background region minimally while predicting the target object's position, and the rotated bounding box could be used as the orientation detection finger knuckle. The subsequent finger knuckle matching process also needs to use the finger knuckle orientation information for the correction operation.


\section{Contactless Finger Knuckle Recognition}

\section{Background of Finger Knuckle Recognition}
The finger knuckle recognition algorithm has attracted much attention and input from researchers, from 2D to 3D. As a result, the corresponding algorithms have achieved high accuracy and efficiency and have been able to cope with differences in finger knuckle features due to ambient light and slight deformation of the finger knuckles. These feature extraction algorithms have been able to extract fairly robust features. These methods use different feature description operators for different feature extraction methods, so the corresponding matching methods also vary. Even a combination of multiple matching methods can be used to enhance the high accuracy of the matching results, such as using both 2D images and 3D images reconstructed from 2D images for extracting finger knuckle features and matching for finger knuckle recognition \cite{cheng2019contactless}. 

There are generally two main types of recognition algorithms for recognition algorithms: one is holistic-based, and the other is local feature-based. For holistic-based, the entire image information in the ROI region is used, and for local feature-based, in short, the domain feature information of the pixels is used. The broad category can be divided into subspace and spectral representation methods for holistic-based \cite{yang2011finger}, \cite{neware2013finger}, \cite{meraoumia2011fusion}. Sub-space methods are generally used for data dimensionality reduction and noise reduction \cite{zhang2006biometric}, such as the use of principal component analysis to reduce the dimensionality of multidimensional data. In contrast to spectral representation methods, image space transformation can be performed as well as image feature enhancement and correlation coefficients for feature extraction \cite{hennings2005verification}. For example, using the Fourier transform to transform the image from the spatial domain to frequency domain information can process frequency information that cannot be processed in the spatial domain.

For global information, algorithms exist relatively for processing local features, which are called local feature-based approaches. For the processing of local information, there are many algorithms, including, for example, extracting information about the gradient of the image edges, obtaining the boundary points, using other edge extraction algorithms such as Hough change. There are also coding-based methods and texture description methods. According to this classification, the method used in this paper uses local feature-based methods and is inspired by coding-based methods for extraction and matching methods. For example, a 1D log-Gabor filter was used to extract the finger knuckles' features and for the matching phase, the hamming distance was used here for the matching score calculation since it is a local feature-based method \cite{meraoumiapersonal}. Alternatively, a 2D Gabor filter is used to extract the domain orientation information features of the finger knuckles, and an angular distance calculation is used to calculate the similarity between the different features for the matching score \cite{mehrotra1992gabor}. High recognition accuracy has been achieved for these matching algorithms, even up to $98.67\%$ \cite{yang2011finger}.


\section{Contactless Finger Knuckle Feature Recognition Based on Deep Neural Network}
Although good accuracy has been achieved for finger knuckle recognition algorithms, as mentioned at the beginning of the paper, they have some limitations for real contactless or real application scenarios. For example, they cannot cope with various complex or contactless sampling scenarios, or they cannot cope with changes in finger knuckle characteristics due to environmental changes, resulting in poor matching accuracy. However, there is no lack of research on contactless scenes. For example, there has been a study to deal with the deformation problem caused by knuckle bending in contactless scenes, and the paper \cite{kumar2019toward} first matches on two images for a selected fixed number $32*32$ of point pairs for coping with the deformation problem and then uses local feature descriptors on each point pair for matching. Even early work considered application scenario using cell phones for finger knuckle recognition \cite{cheng2012contactless}, but for finger knuckle segmentation using fixed finger position in the centre of the image is not very convenient for the user, for recognition phase log-Gabor is used for feature extraction, and Hamming distance is used for matching. 

How to efficiently perform finger knuckle matching, as the various matching algorithms presented in the previous paper, we can use to the application scenario of this paper. However, these algorithms have a source of a problem that these algorithms have to keep changing their corresponding filters and even the corresponding detection parameters under the actual application scenario, or a slight change of the scenario \cite{zheng20163d}. The contactless matching accuracy will be degraded for the contactless scenario if the finger knuckle recognition algorithm verified in the contactless scenario is used directly.